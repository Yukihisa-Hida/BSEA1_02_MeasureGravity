\documentclass{jarticle}

\title{重力加速度の測定}
\author{2511198 肥田幸久}
\date{2025年5月4日}

\begin{document}
\maketitle

\section{目的}

本実験では,ボルダの振り子を用いて精密に測定した振り子の周期から,電気通信大学における重力加速度の値を4桁の精度で測定する.


\section{原理}

\subsection{重力加速度}

地球を球形と仮定し,質量を$M$,半径を$R$,万有引力定数を$G$とすると,地球上の質量$m$の物体に働く重力の大きさ$mg$は
\begin{equation}
  mg=GMm/R^2
\end{equation}
と表され,重力加速度$g$は
\begin{equation}
  g=GM/R^2
\end{equation}
と表される.また,
\begin{itemize}
  \item $G=6.674\times10^{-11}\,\mathrm{N\cdot m^2/kg^2}$
  \item $M=5.972\times10^{24}\,\mathrm{kg}$,
  \item $R=6.378\times10^6\,\mathrm{m}$,
\end{itemize}
を代入して計算すると
\begin{equation}
  g=9.798\,\mathrm{m/s^2}
\end{equation}
を得る.
したがって重力加速度のだいたいの大きさは$g=9.8\,\mathrm{m/s^2}$である.

\subsection{振り子の周期}

単振り子の振動の周期は重力加速度と関係している.振り子の長さを$h$とすると,その周期$T$は
\begin{equation}
  T=2\pi\sqrt{\frac{h}{g}}
\end{equation}
で表される.
この式は,振り子のおもりと振動の振幅が小さい場合の近似式であるが,この式を使えば振り子の周期$T$を測ることで重力加速度$g$は
\begin{equation}
  g=\frac{4\pi^2h}{T^2}
\end{equation}
と求めることができる.

しかし,この式で重力加速度の値を4桁の精度で求めることは難しい.
仮に振り子の長さを$h=1\,\mathrm{m}$とすると,周期は約2秒となる.
式(5)中の$h$を4桁の精度で求めるためには,振り子の長さを不確かさ$1\mathrm{mm}$以内で測る必要があるが,これは容易である.
それに対して,式(5)中の$T^2$を4桁の精度で求めるためには,30周期をストップウォッチで測る場合には時間測定の不確かさを0.06秒以内,60周期の場合にも0.12秒以内にする必要があるが,これは容易ではない.

この例からわかるように,$g$を精密に測るためには周期をもっと精度よく測定する必要がある,

\section{方法}

\section{結果}

\section{考察}

\end{document}